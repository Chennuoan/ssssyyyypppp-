%h%%%%%%%%%%%%%%%%%%%%%%%%%%%%%%%%%%%%%%%%%%%%
%LaTeX 简短的论文期刊模板                   
%%%%%%%%%%%%%%%%%%%%%%%%%%%%%%%%%%%%%%%%%%%%%
\documentclass[a4paper,onecolumn,twoside]{article}
%%%%%%%%%%%%%%%%%%%%%%%%%%%%%%%%%%%%%%%%%%%%%
%需要加载的包
%%%%%%%%%%%%%%%%%%%%%%%%%%%%%%%%%%%%%%%%%%%%%
\usepackage{xeCJK}
\usepackage{fontspec, xunicode, xltxtra}
\usepackage{ifthen}%这个宏包提供逻辑判断命令
\usepackage{fancyhdr}
\usepackage[marginal]{footmisc}%引入宏包,使得脚注首行不缩进。
\usepackage{prettyref}
\usepackage{indentfirst} 
\usepackage{lastpage}                                           
\usepackage{amsmath,amsfonts,amssymb}    % EPS 图片支持
\usepackage{graphicx}
\usepackage[a4paper,hmargin=0.0cm,vmargin=0.0cm,columnsep=2.02em]{geometry}
\usepackage{multirow}
%\usepackage{color}
\usepackage{subfigure}   % 使用子图形
%\usepackage{1-in-2}
\usepackage[labelfont=bf]{caption}
\usepackage{booktabs}
\usepackage{layout}
\usepackage{epstopdf}
\usepackage{multicol}
\usepackage[original]{abstract}%摘要分栏的宏包
\usepackage{titlesec} %其中 center 可使标题居中,还可设为 raggedleft (居左,默认),  
\usepackage{titletoc}%要调整章节标题在目录页中的格式,可以用titletoc宏包 title of contents  
%\usepackage{bibspacing}
%\setlength{\bibspacing}{\baselineskip}
%\usepackage{gbt-7714-2015}
%\usepackage[super,square,comma,sort&compress]{natbib}
\usepackage[numbers,sort&compress]{natbib}
\setlength{\bibsep}{0.0ex}
%\usepackage[hyperref=true,backend=biber,sorting=none,backref=true]{biblatex}
%\usepackage{filecontents}
%\addbibresource{thesis-ref.bib}
\newcommand{\tabincell}[2]{\begin{tabular}{@{}#1@{}}#2\end{tabular}}
%%%%%%%%%%%%%%%%%%%%%%%%%%%%%%%%%%%%%%%%%%%%%
\geometry{left=1.6cm,right=1.8cm,top=2.0cm,bottom=1.7cm,headheight=0.6cm}
%%%%%%%%%%%%%%%%%%%%%%%%%%%%%%%%%%%%%%%%%%%%%
%字体设置
%%%%%%%%%%%%%%%%%%%%%%%%%%%%%%%%%%%%%%%%%%%%%
%---------------xeCjk设置中文字体--------------
\setmainfont{Times New Roman}
\setCJKmainfont[ItalicFont={KaiTi}, BoldFont={SimHei}]{SimSun}
\newcommand{\song}{\CJKfamily{song}}        % 宋体   (windows自带simsun.ttf)  

\setCJKfamilyfont{xs}{Nsimsun}             %新宋体 xs  
\newcommand{\xs}{\CJKfamily{xs}}  
\setCJKfamilyfont{kai}{KaiTi}       %楷体2312  kai  
\newcommand{\kai}{\CJKfamily{kai}} 
\setCJKfamilyfont{yh}{Microsoft YaHei} %微软雅黑 yh  
\newcommand{\yh}{\CJKfamily{yh}} 
\setCJKfamilyfont{hei}{SimHei}               %黑体  hei  
\newcommand{\hei}{\CJKfamily{hei}}       % 黑体   (Windows自带simhei.ttf)  
\setCJKfamilyfont{msunicode}{Arial Unicode MS}            %Arial Unicode MS: msunicode  
\newcommand{\msunicode}{\CJKfamily{msunicode}}  

\setCJKfamilyfont{li}{LiSu}                    %隶书  li  
\newcommand{\li}{\CJKfamily{li}}  
\setCJKfamilyfont{yy}{YouYuan}     %幼圆  yy  
\newcommand{\yy}{\CJKfamily{yy}}  
\setCJKfamilyfont{xm}{MingLiU}                %细明体  xm  
\newcommand{\xm}{\CJKfamily{xm}}  
\setCJKfamilyfont{xxm}{PMingLiU}     %新细明体  xxm  
\newcommand{\xxm}{\CJKfamily{xxm}}
\setCJKfamilyfont{hwsong}{STSong}    %华文宋体  hwsong  
\newcommand{\hwsong}{\CJKfamily{hwsong}}  
\setCJKfamilyfont{hwzs}{STZhongsong}  %华文中宋  hwzs  
\newcommand{\hwzs}{\CJKfamily{hwzs}} 
\setCJKfamilyfont{hwfs}{STFangsong}      %华文仿宋  hwfs  
\newcommand{\hwfs}{\CJKfamily{hwfs}} 
\setCJKfamilyfont{hwxh}{STXihei}          %华文细黑  hwxh  
\newcommand{\hwxh}{\CJKfamily{hwxh}} 
\setCJKfamilyfont{hwl}{STLiti}                  %华文隶书  hwl  
\newcommand{\hwl}{\CJKfamily{hwl}}  
\setCJKfamilyfont{hwxw}{STXinwei}          %华文新魏  hwxw  
\newcommand{\hwxw}{\CJKfamily{hwxw}} 
\setCJKfamilyfont{hwk}{STKaiti}              %华文楷体  hwk  
\newcommand{\hwk}{\CJKfamily{hwk}}  
\setCJKfamilyfont{hwxk}{STXingkai}      %华文行楷  hwxk  
\newcommand{\hwxk}{\CJKfamily{hwxk}} 
\setCJKfamilyfont{hwcy}{STCaiyun}           %华文彩云 hwcy  
\newcommand{\hwcy}{\CJKfamily{hwcy}} 
\setCJKfamilyfont{hwhp}{STHupo}           %华文琥珀   hwhp  
\newcommand{\hwhp}{\CJKfamily{hwhp}} 
\setCJKfamilyfont{fzyao}{FZYaoTi}              %方正姚体  fzy  
\newcommand{\fzyao}{\CJKfamily{fzyao}} 
\setCJKfamilyfont{fzshu}{FZShuTi}                                  %方正舒体 fzshu  
\newcommand{\fzshu}{\CJKfamily{fzshu}} 

%------------------------------设置字体大小------------------------%  
\newcommand{\chuhao}{\fontsize{42pt}{\baselineskip}\selectfont}     %初号  
\newcommand{\xiaochuhao}{\fontsize{36pt}{\baselineskip}\selectfont} %小初号  
\newcommand{\yihao}{\fontsize{28pt}{\baselineskip}\selectfont}      %一号  
\newcommand{\erhao}{\fontsize{21pt}{\baselineskip}\selectfont}      %二号  
\newcommand{\xiaoerhao}{\fontsize{18pt}{\baselineskip}\selectfont}  %小二号  
\newcommand{\sanhao}{\fontsize{15.75pt}{\baselineskip}\selectfont}  %三号  
\newcommand{\sihao}{\fontsize{14pt}{\baselineskip}\selectfont}%     四号  
\newcommand{\xiaosihao}{\fontsize{12pt}{\baselineskip}\selectfont}  %小四号  
\newcommand{\wuhao}{\fontsize{10.5pt}{\baselineskip}\selectfont}    %五号  
\newcommand{\xiaowuhao}{\fontsize{9pt}{\baselineskip}\selectfont}   %小五号  
\newcommand{\liuhao}{\fontsize{7.875pt}{\baselineskip}\selectfont}  %六号  
\newcommand{\qihao}{\fontsize{5.25pt}{\baselineskip}\selectfont}    %七号  

%%%%%%%%%%%%%%%%%%%%%%%%%%%%%%%%%%%%%%%%%%%%%
%页眉页脚设置
%%%%%%%%%%%%%%%%%%%%%%%%%%%%%%%%%%%%%%%%%%%%%
\fancypagestyle{plain}{
\fancyhf{}
\setboolean{first}{false}
\lhead{第~XX~卷\quad 第~X~期\\
\scriptsize{XXXX~年~XX~月}}
\chead{\centering{X~~X~~X~~X\\
\scriptsize{\textbf{Jourinal of *****}}}}
\rhead{Vol. XX, No. XX\\
\scriptsize{May, 2017}}
\lfoot{}
\cfoot{}
\rfoot{}}

%%%%%%%%%%%%%%%%%%%%%%%%%%%%%%%%%%%%%%%%%%%%%%%%%%%%%%%%%%%%%%%%
% 首页后根据奇偶页不同设置页眉页脚
% R,C,L分别代表左中右,O,E代表奇偶页
%%%%%%%%%%%%%%%%%%%%%%%%%%%%%%%%%%%%%%%%%%%%%%%%%%%%%%%%%%%%%%%%
\newboolean{first}%定义一个布尔变量用于判断是否为首页
\setboolean{first}{true}%设定fist变量初值为true
\pagestyle{fancy}
\fancyhf{}
\fancyhead[RE]{第~XX~卷}
\fancyhead[CE]{X~~X~~X~~X}
\fancyhead[LE,RO]{\thepage}
\fancyhead[CO]{罗伟华: XXXXXXXXXXXXXXX}
\fancyhead[LO]{第~X~期}
\lfoot{}
\cfoot{}
\rfoot{}

\renewcommand{\thefootnote}{}%脚注不编号
\newcommand{\makefirstpageheadrule}{%定义首页页眉线绘制命令,这里为等宽双线
\makebox[0pt][l]{\rule[0.55\baselineskip]{\headwidth}{0.4pt}}%
\rule[0.7\baselineskip]{\headwidth}{0.4pt}}
\newcommand{\makeheadrule}{%定义正文页页眉线绘制命令,单线
\rule[0.7\baselineskip]{\headwidth}{0.4pt}}
\renewcommand{\headrule}{%重定义headrule命令
\ifthenelse{\boolean{first}}{\makeheadrule}{\makefirstpageheadrule} }
%%%%%%%%%%%%%%%%%%%%%%%%%%%%%%%%%%%%%%%%%%%%%
%各种距离设置
%%%%%%%%%%%%%%%%%%%%%%%%%%%%%%%%%%%%%%%%%%%%%
%-----------------页眉相关-----------------
\setlength{\topskip}{10.pt}
\setlength{\headsep}{-0.pt}

%-----------------页面布局-----------------
\geometry{left=1.6cm,right=1.8cm,top=2.0cm,bottom=1.7cm}

%-----------------分栏与正文距离-----------
\setlength{\multicolsep}{0.0pt plus 0.0pt minus 0.0pt}% 50% of original
%-----------------题目与正文-----------------
\titleformat{\section}{\normalfont\Large\bfseries}{\thesection}{1em}{}
\titlespacing*{\section} {0pt}{0pt plus 0pt minus .0pt}{6pt plus .0pt}
\titlespacing*{\subsection} {0pt}{0ex plus 0ex minus .0ex}{6pt plus .0pt}
\titlespacing*{\subsubsection} {0pt}{0ex plus 0ex minus .0ex}{0pt plus .0pt}
%-----------------段落与行----------------
\setlength{\parskip}{0ex}
\setlength{\parsep}{0ex} %段落间距
\setlength{\lineskip}{0ex}
\renewcommand{\baselinestretch}{1.3}%设置行间距

% 段间距:\lineskip + \parskip
% 行间距:\lineskip = \baselineskip * \baselinestretch

%----------------调整列表行间距---------------
\usepackage{paralist} 
\let\itemize\compactitem 
\let\enditemize\endcompactitem 
\let\enumerate\compactenum 
\let\endenumerate\endcompactenum 
\let\description\compactdesc 
\let\enddescription\endcompactdesc
%---------------图表caption-------------------
\setlength{\abovecaptionskip}{0pt}%修改Caption距离
\setlength{\belowcaptionskip}{-2pt}

%------------------ 数学公式设置 -----------------
\setlength{\abovedisplayskip}{3pt}     %公式前的距离
\setlength{\belowdisplayskip}{3pt}     %公式后面的距离
%\setlength{\arraycolsep}{2pt}   %在一个array中列之间的空白长度, 因为原来的太宽了

%\makeatletter
%\renewcommand\normalsize{%
%   \@setfontsize\normalsize\@xpt\@xiipt
%   \abovedisplayskip 6\p@ \@plus0\p@ \@minus0\p@
%   \abovedisplayshortskip \z@ \@plus0\p@
%   \belowdisplayshortskip 6\p@ \@plus0\p@ \@minus0\p@
%   \belowdisplayskip \abovedisplayskip
%   \let\@listi\@listI}
%\makeatother

%%%%%%%%%%%%%%%%%%%%%%%%%%%%%%%%%%%%%%%%%%%%%
%标题命令
%%%%%%%%%%%%%%%%%%%%%%%%%%%%%%%%%%%%%%%%%%%%%
%------------------------------标题名称中文化-----------------------------%  
\renewcommand\abstractname{\hei 摘\ 要}  
\renewcommand\refname{\hei 参考文献}  
\renewcommand\figurename{\hei 图}  
\renewcommand\tablename{\hei 表}  
%------------------------------定理名称中文化-----------------------------%  
\newtheorem{dingyi}{\hei 定义~}[section]  
\newtheorem{dingli}{\hei 定理~}[section]  
\newtheorem{yinli}[dingli]{\hei 引理~}  
\newtheorem{tuilun}[dingli]{\hei 推论~}  
\newtheorem{mingti}[dingli]{\hei 命题~}  
\newtheorem{lizi}{{例}}  




%%%%%%%%%%%%%%%%%%%%%%%%%%%%%%%%%%%%%%%%%%%%%
%文章
%%%%%%%%%%%%%%%%%%%%%%%%%%%%%%%%%%%%%%%%%%%%%
\begin{document}  

            % \maketitle%生成标题  
    \thispagestyle{plain}
             \vspace{-3em}
            \begin{center}
              \parbox{\textwidth}{
                \begin{center}
                  {\sanhao{\textbf{××××××论文题目×××××}}}\\
                \vspace{3pt}
                  \vspace{-0.0cm}
                \end{center}
                \begin{center}
                  \textbf{张三$^1$, 李四$^2$}\\[2pt]
                  \small{(1××学院,江西南昌 330031;2中心,江西南昌 330001;)}\\[2pt]
                \end{center}
                {\small{\textbf{摘~要:}这是一个中文模板,在本块添加中文摘要。
                \\
                
                \textbf{关键字:}模板;Latex\\
                \textbf{文献标志码:}A \hspace{4cm} \textbf{中国分类号:}TP18
              }}

              }
            \end{center}


            \begin{center}
              \parbox{\textwidth}{
                \begin{center}
                  {\sanhao{\textbf{××××××××××××Title×××××××××}}}\\
                  \vspace{-0.0cm}
                \end{center}
                \begin{center}
                  \textbf{Zhang San$^1$,Li Si$^2$;}\\[2pt]
                  \small{(1 School of ××××××, Nanchang 330031, Jiangxi;2 The
                  Jiangxi Province ××××××××, Nanchang 330000, Jiangxi)}\\[2pt]
                \end{center}
                {\small{\textbf{Abstract:}\quad This is template of chiese
                Journal Paper
                \\
                \textbf{Keywords}\quad template;Latex}}
              }
            \end{center}

            \noindent{\footnote{\noindent {\textbf{收稿日期}:2017-05-30;\textbf{修回日期}
          :}
          2017-07-16\\ 
          \textbf{基金项目}:国家自然科学基金资助(610****3,8*****69),江西省
          科技计划项目(No. 2011*******87)\\ \textbf{作者简介
          }:
        张三(1992-),男,硕士研究生,主研:智能信息处理、大数据应用;李四,
      男,硕士,高级工程师,主研:气象通讯技术,计算机应用技术;}.
      }

%%%%%%%%%%%%%%%%%%%%%%%%%%%%%%%%%%%%%%%%%%%%%
      %正文部分
%%%%%%%%%%%%%%%%%%%%%%%%%%%%%%%%%%%%%%%%%%%%%
\begin{multicols}{2}
  \section{引言}
  目前,我国饮用水资源处于微污染的阶段,但是在微污染的阶段却存在着不低的风险[1]。我国饮用水水源当中除了常规的偶有超标的污染物指标外,新污染物的检出也是普遍现象。例如,熊小萍[2]等研究了珠江三角洲河流饮用水源的环境内分泌干扰物及其风险,表明了东江东莞段等河流段处于高生态的风险水平。因此,饮用水源需要经过更高质量的处理才能够输送进入居民家中,饮用水纳滤技术便能够满足这一需求。
  \section{正文}     
  \subsection{纳滤定义及原理}
  早在20世纪70年代,人们已经开始对纳滤膜开始了研究,到到20世纪80年代,NF-40和NF-50两种纳滤膜被美国研制出来,纳滤膜技术发展十分迅速,20世纪90年代后相继开发了一系列具有独特分离性能的纳滤膜。\\
  纳滤是一种以压力差为推动力,介于反渗透和超滤之间的截留水中粒径为纳米级颗粒物的一种膜分离技术,主要的核心是分离膜,分离膜通过粒子之间的直径不同来分离污染物[3]。经过半透膜时,不同直径的粒子会进行有选择的筛选,通过一个动力例如压力差,浓度差等提供驱动力,留下半径比膜孔大的粒子,通过半径较小的粒子[3]。
    \subsection{纳滤技术特点} 
    \subsubsection{孔径与截留能力}
  纳滤膜的孔径介于反渗透膜和超滤膜之间,对二价和多价离子及分子量在200~1000之间的有机物有较高的脱除性能,同时能够使单价粒子通过,这个特性使得纳滤膜在饮用水处理时不仅能够去除饮用水中的污染物质,而且可以保留对人体有益的部分矿物质和微量元素[4]。
    \subsubsection{耐污染性}
  纳滤膜具有比较高的耐污染性,纳滤膜的耐污染性能够在酸、碱、高温的恶劣条件下运作,并且膜耐受的条件范围宽[5]。这保证了纳滤膜在长期使用过程中的稳定性和可靠性。
  \subsection{纳滤技术在饮用水中应用}
    \subsubsection{水软化}
  纳滤技术可以用于苦咸水软化当中,我国盐碱地区以及石灰溶地区,水质硬度高达700 mg/L,超标十分严重,上述地区的水体需要软化处理。张显球[6]等阐述了纳滤膜的软水原理,软水用纳滤膜的性能与选用原则,分析了纳滤膜软水法的经济性,并且指出纳滤膜软水法是一种先进且经济的水软化技术,可以克服反渗透的高能耗以及石灰软化和离子交换软化产生的废渣和废水对环境造成的污染。宋跃飞[7]等针对黄淮地区的苦咸水构建了一套小型纳滤膜法水软化系统,分析了pH、TDS、TOC等进水水质对纳滤膜水软化分离性能的影响。\\
  美国佛罗里达州拥有世界上最大的纳滤脱盐软化装置,规模为3.8万m3/d。国内首套工业化纳滤系统示范工程于1997年在山东长岛南隍城建成投产,处理规模144 m3/d。
    \subsubsection{去除微量有机物}
    孔繁鑫[8]分别比较了3种纳滤膜(HL、NF270和NF90)、1种低压反渗透膜(ESPA1)和2种正渗透膜(ES和NW)对卤乙酸和药物的截留特性,探讨了膜污染对微量有机物截留的影响,还对去除微量有机物的模型进行了预测。秦源[9]等发现多孔的MOFs材料可以为纳滤膜提供丰富的纳米孔道,提升纳滤的水通量,同时增加水中微量有机物的去除率。随着我国经济的发展,居民高质量用水意识的提升,纳滤去除微量有机物在饮用水净化领域的作用日益凸显。
    \subsubsection{纳滤净水机应用}
  纳滤净水机在饮用水处理领域展现出了卓越的优势,效能显著。纳滤净水器设备能有效滤除水中的悬浮杂质、胶体物质以及微量有机物,与此同时,精心保留了对人体健康至关重要的微量矿物质元素。这一独特性能,使得纳滤净水机在供应既安全又健康的饮用水方面,赢得了广泛的认可与青睐。\\
  净水技术涵盖分离与纯化两大核心环节,尽管传统的蒸馏与吸附等分离方法在工业领域内得到了广泛应用,但它们因装置庞大、能耗显著等局限性,而在家用自来水净化处理方面的应用受到了制约。随着材料科学的不断进步,膜分离技术迎来了飞速发展。膜分离技术凭借其常温操作、高效分离、节能环保、设备紧凑以及工艺流程简洁等诸多优势,不仅在海水淡化、纯水制备等传统领域大放异彩,还广泛渗透至环保、化工、医药、食品等多个新兴领域[10]。尤为重要的是,膜分离技术的革新为家庭净水设备的研发与应用提供了坚实的技术支持与实现条件。
  
  \subsection{纳滤技术存在的问题}
    \subsubsection{纳滤膜的污染}
  在纳滤处理饮用水的过程中,水中的有机物和无机物会将膜孔处堆积,造成堵塞,进而造成膜表面的污染。在这一过程中膜的操作压力会上升,并且膜通量下降。\\
  膜污染主要分为两种,一种是可逆污染,另外一种是不可逆污染,可逆污染可以通过水来冲洗,不可逆的污染只能使用化学药剂处理[11]。造成膜污染的污染物主要包括有机污染,无机污染,生物污染和颗粒/胶体污染[11]。
    \subsubsection{浓水及冲洗水的处理}
  纳滤过程中会产生大量浓水,纳滤膜只产生分离的作用,因此大量有机物和无机物在浓水中富集,无法降解或转化。\\
  目前为止,膜的清洗有物理清洗和化学清洗,但是在实际情况中,物理清洗难以保证纳滤膜通量的恢复,难以满足需求,因此大多使用化学方法清洗。尽管清洗对于恢复膜的性能至关重要,然而,不当的清洗剂选择可能会致使清洗成效不尽人意,甚至造成膜的永久性损害,难以修复[11]。针对某些特定的污染物,需要使用专门的清洁剂进行有效清除,然而,这些专业清洁产品的具体成分往往不为外界所知。
    
  \section{结论}
  水质安全是人们普遍关注的饮用水处理领域纳滤技术的前景,其应用目的就是为了水质的显著改善,与现代人对高品质饮用水的迫切需求相准确契合。我们热切期盼能够开发出性能更优的纳滤膜,这类膜不仅通量更高,而且能有效提高水处理效率,在实现水质的深层次净化的同时,还表现出较高的选择性。展望未来,随着科学技术的迅猛发展,纳滤净水技术在保障全球水资源安全,促进社会可持续发展方面,无疑将发挥更加关键的作用,在科学技术的日新月异和市场蓬勃发展的今天。不仅可以为人们提供安全可靠的饮用水源,而且可以为人类社会的可持续发展贡献不可缺少的力量,在保护环境和节约资源方面表现出巨大的潜能。
  
    \renewcommand\refname{参考文献}
    
    \bibliography{thesis-ref}
    \cite{example1}
    \bibliographystyle{gbt7714-2005}
    \bibliography{thesis-ref} 
    \setlength{\itemsep}{- 2mm}
    }
  
  \clearpage
  \end{multicols}
  
  \end{document}
  