%h%%%%%%%%%%%%%%%%%%%%%%%%%%%%%%%%%%%%%%%%%%%%
%LaTeX 简短的论文期刊模板                   
%%%%%%%%%%%%%%%%%%%%%%%%%%%%%%%%%%%%%%%%%%%%%
\documentclass[a4paper,onecolumn,twoside]{article}
%h%%%%%%%%%%%%%%%%%%%%%%%%%%%%%%%%%%%%%%%%%%%%
%LaTeX 简短的论文期刊模板                   
%%%%%%%%%%%%%%%%%%%%%%%%%%%%%%%%%%%%%%%%%%%%%
\documentclass[a4paper,onecolumn,twoside]{article}
%h%%%%%%%%%%%%%%%%%%%%%%%%%%%%%%%%%%%%%%%%%%%%
%LaTeX 简短的论文期刊模板                   
%%%%%%%%%%%%%%%%%%%%%%%%%%%%%%%%%%%%%%%%%%%%%
\documentclass[a4paper,onecolumn,twoside]{article}
%h%%%%%%%%%%%%%%%%%%%%%%%%%%%%%%%%%%%%%%%%%%%%
%LaTeX 简短的论文期刊模板                   
%%%%%%%%%%%%%%%%%%%%%%%%%%%%%%%%%%%%%%%%%%%%%
\documentclass[a4paper,onecolumn,twoside]{article}
\input{template.cls}
\usepackage{gbt7714}
\usepackage{graphicx}
\bibliographystyle{gbt7714-numerical}
%%%%%%%%%%%%%%%%%%%%%%%%%%%%%%%%%%%%%%%%%%%%%
%文章
%%%%%%%%%%%%%%%%%%%%%%%%%%%%%%%%%%%%%%%%%%%%%
\begin{document}  

            % \maketitle%生成标题  
    \thispagestyle{plain}
             \vspace{-3em}
            \begin{center}
              \parbox{\textwidth}{
                \begin{center}
                  {\erhao{\textbf{饮用水纳滤技术综述}}}\\
                \vspace{3pt}
                  \vspace{-0.0cm}
                \end{center}
                \begin{center}
                  \textbf{罗伟华$^1$}\\[2pt]
                  \small{(1中山大学环境科学与工程学院,广州,510006)}\\[2pt]
                \end{center}
                {\small{\textbf{摘~要:}想要提高饮用水质量,必须采用更高效,更高质量的饮用水处理方法,饮用水纳滤技术能够为人们提供安全、可靠的饮用水源,还能在环境保护与资源节约方面展现出巨大的潜力,为人类社会的可持续发展贡献不可或缺的力量。本文将详细介绍饮用水纳滤技术,并提出展望。
                \\
                \textbf{关键字:}纳滤、饮用水、污染\\
                \textbf{文献标志码:}A \hspace{4cm} \textbf{中国分类号:}TP18
              }}

              }
            \end{center}


            \begin{center}
              \parbox{\textwidth}{
                \begin{center}
                  {\sanhao{\textbf{An Overview of Nanofiltration Technology for Drinking Water Treatment}}}\\
                  \vspace{-0.0cm}
                \end{center}
                \begin{center}
                  \textbf{Luo Weihua$^1$}\\[2pt]
                  \small{(1 School of Environmental Science and Engineering, Sun Yat-sen University, Guangzhou, 510006, China)}\\[2pt]
                \end{center}
                {\small{\textbf{Abstract:}\quad In order to improve the quality of drinking water, more efficient and high-quality drinking water treatment methods must be adopted. Nanofiltration technology for drinking water can provide people with safe and reliable drinking water sources, and also demonstrate great potential in environmental protection and resource conservation, contributing an indispensable force to the sustainable development of human society. This article will provide a detailed introduction to nanofiltration technology for drinking water and propose prospects.
                \\
                \textbf{Keywords}\quad nanofiltration, drinking water, pollution}}
              }
            \end{center}

            \noindent{\footnote{\noindent {\textbf{收稿日期}:2025-01-18;\textbf{修回日期}
          :}
          \\ 
           \textbf{作者简介
          }:
        罗伟华(2002-),男......}.
      }

%%%%%%%%%%%%%%%%%%%%%%%%%%%%%%%%%%%%%%%%%%%%%
      %正文部分
%%%%%%%%%%%%%%%%%%%%%%%%%%%%%%%%%%%%%%%%%%%%%
% \begin{multicols}{2}
  \section{引言}
  目前,我国饮用水资源处于微污染的阶段,但是在微污染的阶段却存在着不低的风险\cite{LiuZhongGuoShuiYuanWeiWuRanXianZhuangJiFengXianFangKongJianYi2021}。我国饮用水水源当中除了常规的偶有超标的污染物指标外,新污染物的检出也是普遍现象。例如,熊小萍\cite{XiongZhuJiangSanJiaoZhouHeLiuYinYongShuiYuanZhongDeHuanJingNeiFenMiGanRaoWuJiQiFengXian2020}等研究了珠江三角洲河流饮用水源的环境内分泌干扰物及其风险,表明了东江东莞段等河流段处于高生态的风险水平。因此,饮用水源需要经过更高质量的处理才能够输送进入居民家中,饮用水纳滤技术便能够满足这一需求。
  \section{正文}     
  \subsection{纳滤定义及原理}
  早在20世纪70年代,人们已经开始对纳滤膜开始了研究,到到20世纪80年代,NF-40和NF-50两种纳滤膜被美国研制出来,纳滤膜技术发展十分迅速,20世纪90年代后相继开发了一系列具有独特分离性能的纳滤膜。\\
  纳滤是一种以压力差为推动力,介于反渗透和超滤之间的截留水中粒径为纳米级颗粒物的一种膜分离技术,主要的核心是分离膜,分离膜通过粒子之间的直径不同来分离污染物。经过半透膜时,不同直径的粒子会进行有选择的筛选,通过一个动力例如压力差,浓度差等提供驱动力,留下半径比膜孔大的粒子,通过半径较小的粒子\cite{ZhaoMoFenChiJiShuZaiZaiShengShuiZhongDeYingYongJiMoWuRanYanJiuJinZhan2021}。
    \subsection{纳滤技术特点} 
    \subsubsection{孔径与截留能力}
  纳滤膜的孔径介于反渗透膜和超滤膜之间,对二价和多价离子及分子量在200$\~$1000之间的有机物有较高的脱除性能,同时能够使单价粒子通过,这个特性使得纳滤膜在饮用水处理时不仅能够去除饮用水中的污染物质,而且可以保留对人体有益的部分矿物质和微量元素\cite{ZhouNaLuMoShuiChuLiJiShuZaiJingShuiGongYiZhongDeYingYong2023}。
    \subsubsection{耐污染性}
  纳滤膜具有比较高的耐污染性,纳滤膜的耐污染性能够在酸、碱、高温的恶劣条件下运作,并且膜耐受的条件范围宽\cite{LiNaiWuRanXingFuHeNaLuMoZhiBeiJiShuYanJiuJinZhan2016}。这保证了纳滤膜在长期使用过程中的稳定性和可靠性。
  \subsection{纳滤技术在饮用水中应用}
    \subsubsection{水软化}
  纳滤技术可以用于苦咸水软化当中,我国盐碱地区以及石灰溶地区,水质硬度高达700 mg/L,超标十分严重,上述地区的水体需要软化处理。张显球\cite{ZhangMianXiangRuanHuaShuiChuLiDeNaLuMoFenChiJiShu2006}等阐述了纳滤膜的软水原理,软水用纳滤膜的性能与选用原则,分析了纳滤膜软水法的经济性,并且指出纳滤膜软水法是一种先进且经济的水软化技术,可以克服反渗透的高能耗以及石灰软化和离子交换软化产生的废渣和废水对环境造成的污染。宋跃飞等针对黄淮地区的苦咸水构建了一套小型纳滤膜法水软化系统,分析了pH、TDS、TOC等进水水质对纳滤膜水软化分离性能的影响\cite{SongJinShuiShuiZhiDuiNaLuMoKuXianShuiRuanHuaDeFenChiXingNeng2017}。\\
  美国佛罗里达州拥有世界上最大的纳滤脱盐软化装置,规模为3.8万$m^3/d$。国内首套工业化纳滤系统示范工程于1997年在山东长岛南隍城建成投产,处理规模144$m^3/d$。
    \subsubsection{去除微量有机物}
    孔繁鑫分别比较了3种纳滤膜(HL、NF270和NF90)、1种低压反渗透膜(ESPA1)和2种正渗透膜(ES和NW)对卤乙酸和药物的截留特性,探讨了膜污染对微量有机物截留的影响,还对去除微量有机物的模型进行了预测\cite{KongShenTouMoDuiWeiLiangYouJiWuDeQuChuYuChuanZhiTeXingYanJiu2016}。秦源等发现多孔的MOFs材料可以为纳滤膜提供丰富的纳米孔道,提升纳滤的水通量,同时增加水中微量有机物的去除率\cite{QinMOFsGaiXingNaLuMoQuChuYinYongShuiZhongWeiLiangYouJiWuJinZhan2022}。随着我国经济的发展,居民高质量用水意识的提升,纳滤去除微量有机物在饮用水净化领域的作用日益凸显。
    \subsubsection{纳滤净水机应用}
  纳滤净水机在饮用水处理领域展现出了卓越的优势,效能显著。纳滤净水器设备能有效滤除水中的悬浮杂质、胶体物质以及微量有机物,与此同时,精心保留了对人体健康至关重要的微量矿物质元素。这一独特性能,使得纳滤净水机在供应既安全又健康的饮用水方面,赢得了广泛的认可与青睐。\\
  净水技术涵盖分离与纯化两大核心环节,尽管传统的蒸馏与吸附等分离方法在工业领域内得到了广泛应用,但它们因装置庞大、能耗显著等局限性,而在家用自来水净化处理方面的应用受到了制约。随着材料科学的不断进步,膜分离技术迎来了飞速发展。膜分离技术凭借其常温操作、高效分离、节能环保、设备紧凑以及工艺流程简洁等诸多优势,不仅在海水淡化、纯水制备等传统领域大放异彩,还广泛渗透至环保、化工、医药、食品等多个新兴领域\cite{QinMoFenChiJiShuZaiJiaYongJingShuiJiZhongDeYingYong2018}。尤为重要的是,膜分离技术的革新为家庭净水设备的研发与应用提供了坚实的技术支持与实现条件。
  
  \subsection{纳滤技术存在的问题}
    \subsubsection{纳滤膜的污染}
  在纳滤处理饮用水的过程中,水中的有机物和无机物会将膜孔处堆积,造成堵塞,进而造成膜表面的污染。在这一过程中膜的操作压力会上升,并且膜通量下降。\\
  膜污染主要分为两种,一种是可逆污染,另外一种是不可逆污染,可逆污染可以通过水来冲洗,不可逆的污染只能使用化学药剂处理。造成膜污染的污染物主要包括有机污染,无机污染,生物污染和颗粒/胶体污染\cite{ZhaoMoFenChiJiShuZaiZaiShengShuiZhongDeYingYongJiMoWuRanYanJiuJinZhan2021}。
    \subsubsection{浓水及冲洗水的处理}
  纳滤过程中会产生大量浓水,纳滤膜只产生分离的作用,因此大量有机物和无机物在浓水中富集,无法降解或转化。\\
  目前为止,膜的清洗有物理清洗和化学清洗,但是在实际情况中,物理清洗难以保证纳滤膜通量的恢复,难以满足需求,因此大多使用化学方法清洗。尽管清洗对于恢复膜的性能至关重要,然而,不当的清洗剂选择可能会致使清洗成效不尽人意,甚至造成膜的永久性损害,难以修复\cite{ZhaoMoFenChiJiShuZaiZaiShengShuiZhongDeYingYongJiMoWuRanYanJiuJinZhan2021}。针对某些特定的污染物,需要使用专门的清洁剂进行有效清除,然而,这些专业清洁产品的具体成分往往不为外界所知。
    
  \section{结论}
  水质安全是人们普遍关注的饮用水处理领域纳滤技术的前景,其应用目的就是为了水质的显著改善,与现代人对高品质饮用水的迫切需求相准确契合。我们热切期盼能够开发出性能更优的纳滤膜,这类膜不仅通量更高,而且能有效提高水处理效率,在实现水质的深层次净化的同时,还表现出较高的选择性。展望未来,随着科学技术的迅猛发展,纳滤净水技术在保障全球水资源安全,促进社会可持续发展方面,无疑将发挥更加关键的作用,在科学技术的日新月异和市场蓬勃发展的今天。不仅可以为人们提供安全可靠的饮用水源,而且可以为人类社会的可持续发展贡献不可缺少的力量,在保护环境和节约资源方面表现出巨大的潜能。
  
  \renewcommand\refname{参考文献}
  
  \xiaowuhao{
  \bibliography{references_luowei} 
  \setlength{\itemsep}{- 2mm}
}


% \end{multicols}

\end{document}


\usepackage{gbt7714}
\usepackage{graphicx}
\bibliographystyle{gbt7714-numerical}
%%%%%%%%%%%%%%%%%%%%%%%%%%%%%%%%%%%%%%%%%%%%%
%文章
%%%%%%%%%%%%%%%%%%%%%%%%%%%%%%%%%%%%%%%%%%%%%
\begin{document}  

            % \maketitle%生成标题  
    \thispagestyle{plain}
             \vspace{-3em}
            \begin{center}
              \parbox{\textwidth}{
                \begin{center}
                  {\erhao{\textbf{饮用水纳滤技术综述}}}\\
                \vspace{3pt}
                  \vspace{-0.0cm}
                \end{center}
                \begin{center}
                  \textbf{罗伟华$^1$}\\[2pt]
                  \small{(1中山大学环境科学与工程学院,广州,510006)}\\[2pt]
                \end{center}
                {\small{\textbf{摘~要:}想要提高饮用水质量,必须采用更高效,更高质量的饮用水处理方法,饮用水纳滤技术能够为人们提供安全、可靠的饮用水源,还能在环境保护与资源节约方面展现出巨大的潜力,为人类社会的可持续发展贡献不可或缺的力量。本文将详细介绍饮用水纳滤技术,并提出展望。
                \\
                \textbf{关键字:}纳滤、饮用水、污染\\
                \textbf{文献标志码:}A \hspace{4cm} \textbf{中国分类号:}TP18
              }}

              }
            \end{center}


            \begin{center}
              \parbox{\textwidth}{
                \begin{center}
                  {\sanhao{\textbf{An Overview of Nanofiltration Technology for Drinking Water Treatment}}}\\
                  \vspace{-0.0cm}
                \end{center}
                \begin{center}
                  \textbf{Luo Weihua$^1$}\\[2pt]
                  \small{(1 School of Environmental Science and Engineering, Sun Yat-sen University, Guangzhou, 510006, China)}\\[2pt]
                \end{center}
                {\small{\textbf{Abstract:}\quad In order to improve the quality of drinking water, more efficient and high-quality drinking water treatment methods must be adopted. Nanofiltration technology for drinking water can provide people with safe and reliable drinking water sources, and also demonstrate great potential in environmental protection and resource conservation, contributing an indispensable force to the sustainable development of human society. This article will provide a detailed introduction to nanofiltration technology for drinking water and propose prospects.
                \\
                \textbf{Keywords}\quad nanofiltration, drinking water, pollution}}
              }
            \end{center}

            \noindent{\footnote{\noindent {\textbf{收稿日期}:2025-01-18;\textbf{修回日期}
          :}
          \\ 
           \textbf{作者简介
          }:
        罗伟华(2002-),男......}.
      }

%%%%%%%%%%%%%%%%%%%%%%%%%%%%%%%%%%%%%%%%%%%%%
      %正文部分
%%%%%%%%%%%%%%%%%%%%%%%%%%%%%%%%%%%%%%%%%%%%%
% \begin{multicols}{2}
  \section{引言}
  目前,我国饮用水资源处于微污染的阶段,但是在微污染的阶段却存在着不低的风险\cite{LiuZhongGuoShuiYuanWeiWuRanXianZhuangJiFengXianFangKongJianYi2021}。我国饮用水水源当中除了常规的偶有超标的污染物指标外,新污染物的检出也是普遍现象。例如,熊小萍\cite{XiongZhuJiangSanJiaoZhouHeLiuYinYongShuiYuanZhongDeHuanJingNeiFenMiGanRaoWuJiQiFengXian2020}等研究了珠江三角洲河流饮用水源的环境内分泌干扰物及其风险,表明了东江东莞段等河流段处于高生态的风险水平。因此,饮用水源需要经过更高质量的处理才能够输送进入居民家中,饮用水纳滤技术便能够满足这一需求。
  \section{正文}     
  \subsection{纳滤定义及原理}
  早在20世纪70年代,人们已经开始对纳滤膜开始了研究,到到20世纪80年代,NF-40和NF-50两种纳滤膜被美国研制出来,纳滤膜技术发展十分迅速,20世纪90年代后相继开发了一系列具有独特分离性能的纳滤膜。\\
  纳滤是一种以压力差为推动力,介于反渗透和超滤之间的截留水中粒径为纳米级颗粒物的一种膜分离技术,主要的核心是分离膜,分离膜通过粒子之间的直径不同来分离污染物。经过半透膜时,不同直径的粒子会进行有选择的筛选,通过一个动力例如压力差,浓度差等提供驱动力,留下半径比膜孔大的粒子,通过半径较小的粒子\cite{ZhaoMoFenChiJiShuZaiZaiShengShuiZhongDeYingYongJiMoWuRanYanJiuJinZhan2021}。
    \subsection{纳滤技术特点} 
    \subsubsection{孔径与截留能力}
  纳滤膜的孔径介于反渗透膜和超滤膜之间,对二价和多价离子及分子量在200$\~$1000之间的有机物有较高的脱除性能,同时能够使单价粒子通过,这个特性使得纳滤膜在饮用水处理时不仅能够去除饮用水中的污染物质,而且可以保留对人体有益的部分矿物质和微量元素\cite{ZhouNaLuMoShuiChuLiJiShuZaiJingShuiGongYiZhongDeYingYong2023}。
    \subsubsection{耐污染性}
  纳滤膜具有比较高的耐污染性,纳滤膜的耐污染性能够在酸、碱、高温的恶劣条件下运作,并且膜耐受的条件范围宽\cite{LiNaiWuRanXingFuHeNaLuMoZhiBeiJiShuYanJiuJinZhan2016}。这保证了纳滤膜在长期使用过程中的稳定性和可靠性。
  \subsection{纳滤技术在饮用水中应用}
    \subsubsection{水软化}
  纳滤技术可以用于苦咸水软化当中,我国盐碱地区以及石灰溶地区,水质硬度高达700 mg/L,超标十分严重,上述地区的水体需要软化处理。张显球\cite{ZhangMianXiangRuanHuaShuiChuLiDeNaLuMoFenChiJiShu2006}等阐述了纳滤膜的软水原理,软水用纳滤膜的性能与选用原则,分析了纳滤膜软水法的经济性,并且指出纳滤膜软水法是一种先进且经济的水软化技术,可以克服反渗透的高能耗以及石灰软化和离子交换软化产生的废渣和废水对环境造成的污染。宋跃飞等针对黄淮地区的苦咸水构建了一套小型纳滤膜法水软化系统,分析了pH、TDS、TOC等进水水质对纳滤膜水软化分离性能的影响\cite{SongJinShuiShuiZhiDuiNaLuMoKuXianShuiRuanHuaDeFenChiXingNeng2017}。\\
  美国佛罗里达州拥有世界上最大的纳滤脱盐软化装置,规模为3.8万$m^3/d$。国内首套工业化纳滤系统示范工程于1997年在山东长岛南隍城建成投产,处理规模144$m^3/d$。
    \subsubsection{去除微量有机物}
    孔繁鑫分别比较了3种纳滤膜(HL、NF270和NF90)、1种低压反渗透膜(ESPA1)和2种正渗透膜(ES和NW)对卤乙酸和药物的截留特性,探讨了膜污染对微量有机物截留的影响,还对去除微量有机物的模型进行了预测\cite{KongShenTouMoDuiWeiLiangYouJiWuDeQuChuYuChuanZhiTeXingYanJiu2016}。秦源等发现多孔的MOFs材料可以为纳滤膜提供丰富的纳米孔道,提升纳滤的水通量,同时增加水中微量有机物的去除率\cite{QinMOFsGaiXingNaLuMoQuChuYinYongShuiZhongWeiLiangYouJiWuJinZhan2022}。随着我国经济的发展,居民高质量用水意识的提升,纳滤去除微量有机物在饮用水净化领域的作用日益凸显。
    \subsubsection{纳滤净水机应用}
  纳滤净水机在饮用水处理领域展现出了卓越的优势,效能显著。纳滤净水器设备能有效滤除水中的悬浮杂质、胶体物质以及微量有机物,与此同时,精心保留了对人体健康至关重要的微量矿物质元素。这一独特性能,使得纳滤净水机在供应既安全又健康的饮用水方面,赢得了广泛的认可与青睐。\\
  净水技术涵盖分离与纯化两大核心环节,尽管传统的蒸馏与吸附等分离方法在工业领域内得到了广泛应用,但它们因装置庞大、能耗显著等局限性,而在家用自来水净化处理方面的应用受到了制约。随着材料科学的不断进步,膜分离技术迎来了飞速发展。膜分离技术凭借其常温操作、高效分离、节能环保、设备紧凑以及工艺流程简洁等诸多优势,不仅在海水淡化、纯水制备等传统领域大放异彩,还广泛渗透至环保、化工、医药、食品等多个新兴领域\cite{QinMoFenChiJiShuZaiJiaYongJingShuiJiZhongDeYingYong2018}。尤为重要的是,膜分离技术的革新为家庭净水设备的研发与应用提供了坚实的技术支持与实现条件。
  
  \subsection{纳滤技术存在的问题}
    \subsubsection{纳滤膜的污染}
  在纳滤处理饮用水的过程中,水中的有机物和无机物会将膜孔处堆积,造成堵塞,进而造成膜表面的污染。在这一过程中膜的操作压力会上升,并且膜通量下降。\\
  膜污染主要分为两种,一种是可逆污染,另外一种是不可逆污染,可逆污染可以通过水来冲洗,不可逆的污染只能使用化学药剂处理。造成膜污染的污染物主要包括有机污染,无机污染,生物污染和颗粒/胶体污染\cite{ZhaoMoFenChiJiShuZaiZaiShengShuiZhongDeYingYongJiMoWuRanYanJiuJinZhan2021}。
    \subsubsection{浓水及冲洗水的处理}
  纳滤过程中会产生大量浓水,纳滤膜只产生分离的作用,因此大量有机物和无机物在浓水中富集,无法降解或转化。\\
  目前为止,膜的清洗有物理清洗和化学清洗,但是在实际情况中,物理清洗难以保证纳滤膜通量的恢复,难以满足需求,因此大多使用化学方法清洗。尽管清洗对于恢复膜的性能至关重要,然而,不当的清洗剂选择可能会致使清洗成效不尽人意,甚至造成膜的永久性损害,难以修复\cite{ZhaoMoFenChiJiShuZaiZaiShengShuiZhongDeYingYongJiMoWuRanYanJiuJinZhan2021}。针对某些特定的污染物,需要使用专门的清洁剂进行有效清除,然而,这些专业清洁产品的具体成分往往不为外界所知。
    
  \section{结论}
  水质安全是人们普遍关注的饮用水处理领域纳滤技术的前景,其应用目的就是为了水质的显著改善,与现代人对高品质饮用水的迫切需求相准确契合。我们热切期盼能够开发出性能更优的纳滤膜,这类膜不仅通量更高,而且能有效提高水处理效率,在实现水质的深层次净化的同时,还表现出较高的选择性。展望未来,随着科学技术的迅猛发展,纳滤净水技术在保障全球水资源安全,促进社会可持续发展方面,无疑将发挥更加关键的作用,在科学技术的日新月异和市场蓬勃发展的今天。不仅可以为人们提供安全可靠的饮用水源,而且可以为人类社会的可持续发展贡献不可缺少的力量,在保护环境和节约资源方面表现出巨大的潜能。
  
  \renewcommand\refname{参考文献}
  
  \xiaowuhao{
  \bibliography{references_luowei} 
  \setlength{\itemsep}{- 2mm}
}


% \end{multicols}

\end{document}


\usepackage{gbt7714}
\usepackage{graphicx}
\bibliographystyle{gbt7714-numerical}
%%%%%%%%%%%%%%%%%%%%%%%%%%%%%%%%%%%%%%%%%%%%%
%文章
%%%%%%%%%%%%%%%%%%%%%%%%%%%%%%%%%%%%%%%%%%%%%
\begin{document}  

            % \maketitle%生成标题  
    \thispagestyle{plain}
             \vspace{-3em}
            \begin{center}
              \parbox{\textwidth}{
                \begin{center}
                  {\erhao{\textbf{饮用水纳滤技术综述}}}\\
                \vspace{3pt}
                  \vspace{-0.0cm}
                \end{center}
                \begin{center}
                  \textbf{罗伟华$^1$}\\[2pt]
                  \small{(1中山大学环境科学与工程学院,广州,510006)}\\[2pt]
                \end{center}
                {\small{\textbf{摘~要:}想要提高饮用水质量,必须采用更高效,更高质量的饮用水处理方法,饮用水纳滤技术能够为人们提供安全、可靠的饮用水源,还能在环境保护与资源节约方面展现出巨大的潜力,为人类社会的可持续发展贡献不可或缺的力量。本文将详细介绍饮用水纳滤技术,并提出展望。
                \\
                \textbf{关键字:}纳滤、饮用水、污染\\
                \textbf{文献标志码:}A \hspace{4cm} \textbf{中国分类号:}TP18
              }}

              }
            \end{center}


            \begin{center}
              \parbox{\textwidth}{
                \begin{center}
                  {\sanhao{\textbf{An Overview of Nanofiltration Technology for Drinking Water Treatment}}}\\
                  \vspace{-0.0cm}
                \end{center}
                \begin{center}
                  \textbf{Luo Weihua$^1$}\\[2pt]
                  \small{(1 School of Environmental Science and Engineering, Sun Yat-sen University, Guangzhou, 510006, China)}\\[2pt]
                \end{center}
                {\small{\textbf{Abstract:}\quad In order to improve the quality of drinking water, more efficient and high-quality drinking water treatment methods must be adopted. Nanofiltration technology for drinking water can provide people with safe and reliable drinking water sources, and also demonstrate great potential in environmental protection and resource conservation, contributing an indispensable force to the sustainable development of human society. This article will provide a detailed introduction to nanofiltration technology for drinking water and propose prospects.
                \\
                \textbf{Keywords}\quad nanofiltration, drinking water, pollution}}
              }
            \end{center}

            \noindent{\footnote{\noindent {\textbf{收稿日期}:2025-01-18;\textbf{修回日期}
          :}
          \\ 
           \textbf{作者简介
          }:
        罗伟华(2002-),男......}.
      }

%%%%%%%%%%%%%%%%%%%%%%%%%%%%%%%%%%%%%%%%%%%%%
      %正文部分
%%%%%%%%%%%%%%%%%%%%%%%%%%%%%%%%%%%%%%%%%%%%%
% \begin{multicols}{2}
  \section{引言}
  目前,我国饮用水资源处于微污染的阶段,但是在微污染的阶段却存在着不低的风险\cite{LiuZhongGuoShuiYuanWeiWuRanXianZhuangJiFengXianFangKongJianYi2021}。我国饮用水水源当中除了常规的偶有超标的污染物指标外,新污染物的检出也是普遍现象。例如,熊小萍\cite{XiongZhuJiangSanJiaoZhouHeLiuYinYongShuiYuanZhongDeHuanJingNeiFenMiGanRaoWuJiQiFengXian2020}等研究了珠江三角洲河流饮用水源的环境内分泌干扰物及其风险,表明了东江东莞段等河流段处于高生态的风险水平。因此,饮用水源需要经过更高质量的处理才能够输送进入居民家中,饮用水纳滤技术便能够满足这一需求。
  \section{正文}     
  \subsection{纳滤定义及原理}
  早在20世纪70年代,人们已经开始对纳滤膜开始了研究,到到20世纪80年代,NF-40和NF-50两种纳滤膜被美国研制出来,纳滤膜技术发展十分迅速,20世纪90年代后相继开发了一系列具有独特分离性能的纳滤膜。\\
  纳滤是一种以压力差为推动力,介于反渗透和超滤之间的截留水中粒径为纳米级颗粒物的一种膜分离技术,主要的核心是分离膜,分离膜通过粒子之间的直径不同来分离污染物。经过半透膜时,不同直径的粒子会进行有选择的筛选,通过一个动力例如压力差,浓度差等提供驱动力,留下半径比膜孔大的粒子,通过半径较小的粒子\cite{ZhaoMoFenChiJiShuZaiZaiShengShuiZhongDeYingYongJiMoWuRanYanJiuJinZhan2021}。
    \subsection{纳滤技术特点} 
    \subsubsection{孔径与截留能力}
  纳滤膜的孔径介于反渗透膜和超滤膜之间,对二价和多价离子及分子量在200$\~$1000之间的有机物有较高的脱除性能,同时能够使单价粒子通过,这个特性使得纳滤膜在饮用水处理时不仅能够去除饮用水中的污染物质,而且可以保留对人体有益的部分矿物质和微量元素\cite{ZhouNaLuMoShuiChuLiJiShuZaiJingShuiGongYiZhongDeYingYong2023}。
    \subsubsection{耐污染性}
  纳滤膜具有比较高的耐污染性,纳滤膜的耐污染性能够在酸、碱、高温的恶劣条件下运作,并且膜耐受的条件范围宽\cite{LiNaiWuRanXingFuHeNaLuMoZhiBeiJiShuYanJiuJinZhan2016}。这保证了纳滤膜在长期使用过程中的稳定性和可靠性。
  \subsection{纳滤技术在饮用水中应用}
    \subsubsection{水软化}
  纳滤技术可以用于苦咸水软化当中,我国盐碱地区以及石灰溶地区,水质硬度高达700 mg/L,超标十分严重,上述地区的水体需要软化处理。张显球\cite{ZhangMianXiangRuanHuaShuiChuLiDeNaLuMoFenChiJiShu2006}等阐述了纳滤膜的软水原理,软水用纳滤膜的性能与选用原则,分析了纳滤膜软水法的经济性,并且指出纳滤膜软水法是一种先进且经济的水软化技术,可以克服反渗透的高能耗以及石灰软化和离子交换软化产生的废渣和废水对环境造成的污染。宋跃飞等针对黄淮地区的苦咸水构建了一套小型纳滤膜法水软化系统,分析了pH、TDS、TOC等进水水质对纳滤膜水软化分离性能的影响\cite{SongJinShuiShuiZhiDuiNaLuMoKuXianShuiRuanHuaDeFenChiXingNeng2017}。\\
  美国佛罗里达州拥有世界上最大的纳滤脱盐软化装置,规模为3.8万$m^3/d$。国内首套工业化纳滤系统示范工程于1997年在山东长岛南隍城建成投产,处理规模144$m^3/d$。
    \subsubsection{去除微量有机物}
    孔繁鑫分别比较了3种纳滤膜(HL、NF270和NF90)、1种低压反渗透膜(ESPA1)和2种正渗透膜(ES和NW)对卤乙酸和药物的截留特性,探讨了膜污染对微量有机物截留的影响,还对去除微量有机物的模型进行了预测\cite{KongShenTouMoDuiWeiLiangYouJiWuDeQuChuYuChuanZhiTeXingYanJiu2016}。秦源等发现多孔的MOFs材料可以为纳滤膜提供丰富的纳米孔道,提升纳滤的水通量,同时增加水中微量有机物的去除率\cite{QinMOFsGaiXingNaLuMoQuChuYinYongShuiZhongWeiLiangYouJiWuJinZhan2022}。随着我国经济的发展,居民高质量用水意识的提升,纳滤去除微量有机物在饮用水净化领域的作用日益凸显。
    \subsubsection{纳滤净水机应用}
  纳滤净水机在饮用水处理领域展现出了卓越的优势,效能显著。纳滤净水器设备能有效滤除水中的悬浮杂质、胶体物质以及微量有机物,与此同时,精心保留了对人体健康至关重要的微量矿物质元素。这一独特性能,使得纳滤净水机在供应既安全又健康的饮用水方面,赢得了广泛的认可与青睐。\\
  净水技术涵盖分离与纯化两大核心环节,尽管传统的蒸馏与吸附等分离方法在工业领域内得到了广泛应用,但它们因装置庞大、能耗显著等局限性,而在家用自来水净化处理方面的应用受到了制约。随着材料科学的不断进步,膜分离技术迎来了飞速发展。膜分离技术凭借其常温操作、高效分离、节能环保、设备紧凑以及工艺流程简洁等诸多优势,不仅在海水淡化、纯水制备等传统领域大放异彩,还广泛渗透至环保、化工、医药、食品等多个新兴领域\cite{QinMoFenChiJiShuZaiJiaYongJingShuiJiZhongDeYingYong2018}。尤为重要的是,膜分离技术的革新为家庭净水设备的研发与应用提供了坚实的技术支持与实现条件。
  
  \subsection{纳滤技术存在的问题}
    \subsubsection{纳滤膜的污染}
  在纳滤处理饮用水的过程中,水中的有机物和无机物会将膜孔处堆积,造成堵塞,进而造成膜表面的污染。在这一过程中膜的操作压力会上升,并且膜通量下降。\\
  膜污染主要分为两种,一种是可逆污染,另外一种是不可逆污染,可逆污染可以通过水来冲洗,不可逆的污染只能使用化学药剂处理。造成膜污染的污染物主要包括有机污染,无机污染,生物污染和颗粒/胶体污染\cite{ZhaoMoFenChiJiShuZaiZaiShengShuiZhongDeYingYongJiMoWuRanYanJiuJinZhan2021}。
    \subsubsection{浓水及冲洗水的处理}
  纳滤过程中会产生大量浓水,纳滤膜只产生分离的作用,因此大量有机物和无机物在浓水中富集,无法降解或转化。\\
  目前为止,膜的清洗有物理清洗和化学清洗,但是在实际情况中,物理清洗难以保证纳滤膜通量的恢复,难以满足需求,因此大多使用化学方法清洗。尽管清洗对于恢复膜的性能至关重要,然而,不当的清洗剂选择可能会致使清洗成效不尽人意,甚至造成膜的永久性损害,难以修复\cite{ZhaoMoFenChiJiShuZaiZaiShengShuiZhongDeYingYongJiMoWuRanYanJiuJinZhan2021}。针对某些特定的污染物,需要使用专门的清洁剂进行有效清除,然而,这些专业清洁产品的具体成分往往不为外界所知。
    
  \section{结论}
  水质安全是人们普遍关注的饮用水处理领域纳滤技术的前景,其应用目的就是为了水质的显著改善,与现代人对高品质饮用水的迫切需求相准确契合。我们热切期盼能够开发出性能更优的纳滤膜,这类膜不仅通量更高,而且能有效提高水处理效率,在实现水质的深层次净化的同时,还表现出较高的选择性。展望未来,随着科学技术的迅猛发展,纳滤净水技术在保障全球水资源安全,促进社会可持续发展方面,无疑将发挥更加关键的作用,在科学技术的日新月异和市场蓬勃发展的今天。不仅可以为人们提供安全可靠的饮用水源,而且可以为人类社会的可持续发展贡献不可缺少的力量,在保护环境和节约资源方面表现出巨大的潜能。
  
  \renewcommand\refname{参考文献}
  
  \xiaowuhao{
  \bibliography{references_luowei} 
  \setlength{\itemsep}{- 2mm}
}


% \end{multicols}

\end{document}


\usepackage{gbt7714}
\usepackage{graphicx}
\bibliographystyle{gbt7714-numerical}
%%%%%%%%%%%%%%%%%%%%%%%%%%%%%%%%%%%%%%%%%%%%%
%文章
%%%%%%%%%%%%%%%%%%%%%%%%%%%%%%%%%%%%%%%%%%%%%
\begin{document}  

            % \maketitle%生成标题  
    \thispagestyle{plain}
             \vspace{-3em}
            \begin{center}
              \parbox{\textwidth}{
                \begin{center}
                  {\erhao{\textbf{饮用水纳滤技术综述}}}\\
                \vspace{3pt}
                  \vspace{-0.0cm}
                \end{center}
                \begin{center}
                  \textbf{罗伟华$^1$}\\[2pt]
                  \small{(1中山大学环境科学与工程学院,广州,510006)}\\[2pt]
                \end{center}
                {\small{\textbf{摘~要:}想要提高饮用水质量,必须采用更高效,更高质量的饮用水处理方法,饮用水纳滤技术能够为人们提供安全、可靠的饮用水源,还能在环境保护与资源节约方面展现出巨大的潜力,为人类社会的可持续发展贡献不可或缺的力量。本文将详细介绍饮用水纳滤技术,并提出展望。
                \\
                \textbf{关键字:}纳滤、饮用水、污染\\
                \textbf{文献标志码:}A \hspace{4cm} \textbf{中国分类号:}TP18
              }}

              }
            \end{center}


            \begin{center}
              \parbox{\textwidth}{
                \begin{center}
                  {\sanhao{\textbf{An Overview of Nanofiltration Technology for Drinking Water Treatment}}}\\
                  \vspace{-0.0cm}
                \end{center}
                \begin{center}
                  \textbf{Luo Weihua$^1$}\\[2pt]
                  \small{(1 School of Environmental Science and Engineering, Sun Yat-sen University, Guangzhou, 510006, China)}\\[2pt]
                \end{center}
                {\small{\textbf{Abstract:}\quad In order to improve the quality of drinking water, more efficient and high-quality drinking water treatment methods must be adopted. Nanofiltration technology for drinking water can provide people with safe and reliable drinking water sources, and also demonstrate great potential in environmental protection and resource conservation, contributing an indispensable force to the sustainable development of human society. This article will provide a detailed introduction to nanofiltration technology for drinking water and propose prospects.
                \\
                \textbf{Keywords}\quad nanofiltration, drinking water, pollution}}
              }
            \end{center}

            \noindent{\footnote{\noindent {\textbf{收稿日期}:2025-01-18;\textbf{修回日期}
          :}
          \\ 
           \textbf{作者简介
          }:
        罗伟华(2002-),男......}.
      }

%%%%%%%%%%%%%%%%%%%%%%%%%%%%%%%%%%%%%%%%%%%%%
      %正文部分
%%%%%%%%%%%%%%%%%%%%%%%%%%%%%%%%%%%%%%%%%%%%%
% \begin{multicols}{2}
  \section{引言}
  目前,我国饮用水资源处于微污染的阶段,但是在微污染的阶段却存在着不低的风险\cite{LiuZhongGuoShuiYuanWeiWuRanXianZhuangJiFengXianFangKongJianYi2021}。我国饮用水水源当中除了常规的偶有超标的污染物指标外,新污染物的检出也是普遍现象。例如,熊小萍\cite{XiongZhuJiangSanJiaoZhouHeLiuYinYongShuiYuanZhongDeHuanJingNeiFenMiGanRaoWuJiQiFengXian2020}等研究了珠江三角洲河流饮用水源的环境内分泌干扰物及其风险,表明了东江东莞段等河流段处于高生态的风险水平。因此,饮用水源需要经过更高质量的处理才能够输送进入居民家中,饮用水纳滤技术便能够满足这一需求。
  \section{正文}     
  \subsection{纳滤定义及原理}
  早在20世纪70年代,人们已经开始对纳滤膜开始了研究,到到20世纪80年代,NF-40和NF-50两种纳滤膜被美国研制出来,纳滤膜技术发展十分迅速,20世纪90年代后相继开发了一系列具有独特分离性能的纳滤膜。\\
  纳滤是一种以压力差为推动力,介于反渗透和超滤之间的截留水中粒径为纳米级颗粒物的一种膜分离技术,主要的核心是分离膜,分离膜通过粒子之间的直径不同来分离污染物。经过半透膜时,不同直径的粒子会进行有选择的筛选,通过一个动力例如压力差,浓度差等提供驱动力,留下半径比膜孔大的粒子,通过半径较小的粒子\cite{ZhaoMoFenChiJiShuZaiZaiShengShuiZhongDeYingYongJiMoWuRanYanJiuJinZhan2021}。
    \subsection{纳滤技术特点} 
    \subsubsection{孔径与截留能力}
  纳滤膜的孔径介于反渗透膜和超滤膜之间,对二价和多价离子及分子量在200$\~$1000之间的有机物有较高的脱除性能,同时能够使单价粒子通过,这个特性使得纳滤膜在饮用水处理时不仅能够去除饮用水中的污染物质,而且可以保留对人体有益的部分矿物质和微量元素\cite{ZhouNaLuMoShuiChuLiJiShuZaiJingShuiGongYiZhongDeYingYong2023}。
    \subsubsection{耐污染性}
  纳滤膜具有比较高的耐污染性,纳滤膜的耐污染性能够在酸、碱、高温的恶劣条件下运作,并且膜耐受的条件范围宽\cite{LiNaiWuRanXingFuHeNaLuMoZhiBeiJiShuYanJiuJinZhan2016}。这保证了纳滤膜在长期使用过程中的稳定性和可靠性。
  \subsection{纳滤技术在饮用水中应用}
    \subsubsection{水软化}
  纳滤技术可以用于苦咸水软化当中,我国盐碱地区以及石灰溶地区,水质硬度高达700 mg/L,超标十分严重,上述地区的水体需要软化处理。张显球\cite{ZhangMianXiangRuanHuaShuiChuLiDeNaLuMoFenChiJiShu2006}等阐述了纳滤膜的软水原理,软水用纳滤膜的性能与选用原则,分析了纳滤膜软水法的经济性,并且指出纳滤膜软水法是一种先进且经济的水软化技术,可以克服反渗透的高能耗以及石灰软化和离子交换软化产生的废渣和废水对环境造成的污染。宋跃飞等针对黄淮地区的苦咸水构建了一套小型纳滤膜法水软化系统,分析了pH、TDS、TOC等进水水质对纳滤膜水软化分离性能的影响\cite{SongJinShuiShuiZhiDuiNaLuMoKuXianShuiRuanHuaDeFenChiXingNeng2017}。\\
  美国佛罗里达州拥有世界上最大的纳滤脱盐软化装置,规模为3.8万$m^3/d$。国内首套工业化纳滤系统示范工程于1997年在山东长岛南隍城建成投产,处理规模144$m^3/d$。
    \subsubsection{去除微量有机物}
    孔繁鑫分别比较了3种纳滤膜(HL、NF270和NF90)、1种低压反渗透膜(ESPA1)和2种正渗透膜(ES和NW)对卤乙酸和药物的截留特性,探讨了膜污染对微量有机物截留的影响,还对去除微量有机物的模型进行了预测\cite{KongShenTouMoDuiWeiLiangYouJiWuDeQuChuYuChuanZhiTeXingYanJiu2016}。秦源等发现多孔的MOFs材料可以为纳滤膜提供丰富的纳米孔道,提升纳滤的水通量,同时增加水中微量有机物的去除率\cite{QinMOFsGaiXingNaLuMoQuChuYinYongShuiZhongWeiLiangYouJiWuJinZhan2022}。随着我国经济的发展,居民高质量用水意识的提升,纳滤去除微量有机物在饮用水净化领域的作用日益凸显。
    \subsubsection{纳滤净水机应用}
  纳滤净水机在饮用水处理领域展现出了卓越的优势,效能显著。纳滤净水器设备能有效滤除水中的悬浮杂质、胶体物质以及微量有机物,与此同时,精心保留了对人体健康至关重要的微量矿物质元素。这一独特性能,使得纳滤净水机在供应既安全又健康的饮用水方面,赢得了广泛的认可与青睐。\\
  净水技术涵盖分离与纯化两大核心环节,尽管传统的蒸馏与吸附等分离方法在工业领域内得到了广泛应用,但它们因装置庞大、能耗显著等局限性,而在家用自来水净化处理方面的应用受到了制约。随着材料科学的不断进步,膜分离技术迎来了飞速发展。膜分离技术凭借其常温操作、高效分离、节能环保、设备紧凑以及工艺流程简洁等诸多优势,不仅在海水淡化、纯水制备等传统领域大放异彩,还广泛渗透至环保、化工、医药、食品等多个新兴领域\cite{QinMoFenChiJiShuZaiJiaYongJingShuiJiZhongDeYingYong2018}。尤为重要的是,膜分离技术的革新为家庭净水设备的研发与应用提供了坚实的技术支持与实现条件。
  
  \subsection{纳滤技术存在的问题}
    \subsubsection{纳滤膜的污染}
  在纳滤处理饮用水的过程中,水中的有机物和无机物会将膜孔处堆积,造成堵塞,进而造成膜表面的污染。在这一过程中膜的操作压力会上升,并且膜通量下降。\\
  膜污染主要分为两种,一种是可逆污染,另外一种是不可逆污染,可逆污染可以通过水来冲洗,不可逆的污染只能使用化学药剂处理。造成膜污染的污染物主要包括有机污染,无机污染,生物污染和颗粒/胶体污染\cite{ZhaoMoFenChiJiShuZaiZaiShengShuiZhongDeYingYongJiMoWuRanYanJiuJinZhan2021}。
    \subsubsection{浓水及冲洗水的处理}
  纳滤过程中会产生大量浓水,纳滤膜只产生分离的作用,因此大量有机物和无机物在浓水中富集,无法降解或转化。\\
  目前为止,膜的清洗有物理清洗和化学清洗,但是在实际情况中,物理清洗难以保证纳滤膜通量的恢复,难以满足需求,因此大多使用化学方法清洗。尽管清洗对于恢复膜的性能至关重要,然而,不当的清洗剂选择可能会致使清洗成效不尽人意,甚至造成膜的永久性损害,难以修复\cite{ZhaoMoFenChiJiShuZaiZaiShengShuiZhongDeYingYongJiMoWuRanYanJiuJinZhan2021}。针对某些特定的污染物,需要使用专门的清洁剂进行有效清除,然而,这些专业清洁产品的具体成分往往不为外界所知。
    
  \section{结论}
  水质安全是人们普遍关注的饮用水处理领域纳滤技术的前景,其应用目的就是为了水质的显著改善,与现代人对高品质饮用水的迫切需求相准确契合。我们热切期盼能够开发出性能更优的纳滤膜,这类膜不仅通量更高,而且能有效提高水处理效率,在实现水质的深层次净化的同时,还表现出较高的选择性。展望未来,随着科学技术的迅猛发展,纳滤净水技术在保障全球水资源安全,促进社会可持续发展方面,无疑将发挥更加关键的作用,在科学技术的日新月异和市场蓬勃发展的今天。不仅可以为人们提供安全可靠的饮用水源,而且可以为人类社会的可持续发展贡献不可缺少的力量,在保护环境和节约资源方面表现出巨大的潜能。
  
  \renewcommand\refname{参考文献}
  
  \xiaowuhao{
  \bibliography{references_luowei} 
  \setlength{\itemsep}{- 2mm}
}


% \end{multicols}

\end{document}

